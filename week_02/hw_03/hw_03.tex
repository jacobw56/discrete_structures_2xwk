\documentclass[12pt, letterpaper]{article}

\usepackage[letterpaper,left=0.7in,right=0.7in,top=\dimexpr15mm+1.5\baselineskip,bottom=0.7in]{geometry}
\usepackage{amsmath}
\usepackage{amsthm}
%\usepackage{pxfonts}
\usepackage[]{newpxtext}
\usepackage[varg,vvarbb]{newpxmath}
\usepackage{inconsolata}
\usepackage[T1]{fontenc}
\usepackage{float}
\usepackage{mathtools}
\usepackage{mathrsfs}
\usepackage{anyfontsize}

\theoremstyle{definition}
\newtheorem{defn}{Definition}
\newtheorem{propt}{Proposition}
\newtheorem{eg}{Example}
\newtheorem{soln}{Solution}
\newtheorem{lemma}{Lemma}
\newtheorem{theorem}{Theorem}

\newenvironment{prop}[1]{%
    \vspace*{0.2in}
    \linebreak
    \begin{minipage}{\linewidth}
    \rule{\textwidth}{2pt}
        \begin{propt}
}
{%
        \end{propt}
    \rule{\textwidth}{2pt}
    \end{minipage}
    \vspace*{0.2in}
    \linebreak
}

\usepackage{enumitem}
\usepackage{graphicx}
\usepackage{multicol}
\usepackage[table]{xcolor}

\usepackage{listings}
\lstset{
    basicstyle=\ttfamily, % Default font
    numbers=left,              % Location of line numbers
    % numberstyle=\tiny,          % Style of line numbers
    % stepnumber=2,              % Margin between line numbers
    numbersep=5pt,              % Margin between line numbers and text
    tabsize=2,                  % Size of tabs
    extendedchars=true,
    breaklines=true,            % Lines will be wrapped
    keywordstyle=\color{red},
    frame=b,
    % keywordstyle=[1]\textbf,
    % keywordstyle=[2]\textbf,
    % keywordstyle=[3]\textbf,
    % keywordstyle=[4]\textbf,   \sqrt{\sqrt{}}
    stringstyle=\color{white}\ttfamily, % Color of strings
    showspaces=false,
    showtabs=false,
    xleftmargin=17pt,
    %framexbottommargin=0.2in
    framexleftmargin=17pt,
    framexrightmargin=5pt,
    framexbottommargin=4pt,
    % backgroundcolor=\color{lightgray},
    showstringspaces=false
}
%\lstloadlanguages{ % Check documentation for further languages ...
     % [Visual]Basic,
     % Pascal,
     % C,
     % C++,
     % XML,
     % HTML,
     %Java
%}
% \DeclareCaptionFont{blue}{\color{blue}} 

% \captionsetup[lstlisting]{singlelinecheck=false, labelfont={blue}, textfont={blue}}
\usepackage{caption}
\DeclareCaptionFont{white}{\color{white}}
\DeclareCaptionFormat{listing}{\colorbox[cmyk]{0.7, 0.7, 0.7,0.01}{\parbox{0.98\textwidth}{\hspace{15pt}#1#2#3}}}
\captionsetup[lstlisting]{
    format=listing, 
    labelfont=white, 
    labelformat=empty,
    textfont=white, 
    singlelinecheck=false, 
    %margin=0pt, 
    font={bf}
    }

\usepackage{fancyhdr}
\setlength{\headheight}{15.2pt}
\pagestyle{fancyplain}
\fancyhf{}
\lhead[Discrete Structures]{Discrete Structures}
\chead[Homework 2]{Homework 2}
%\rhead[\today]{\today}

\newcommand{\gap}[1][0.3in]{
    \vspace*{#1}
    \noindent\noindent
}

\newcommand{\septext}[1][or]{
    \qquad \text{#1} \qquad
}

\newcommand{\septextor}[0]{
    \septext
}

\newcommand{\septextand}[0]{
    \septext[and]
}

\newcommand{\separrow}[0]{
    \qquad \Rightarrow \qquad
}

\usepackage{circuitikz}
\usepackage{pgfplots}
\pgfplotsset{compat=1.18}
%\usepgfplotslibrary{polar}
\usepackage{wrapfig}
\usepackage{setspace}
\usepackage[export]{adjustbox}

\DeclareMathOperator{\arcsinh}{arcsinh}

\linespread{1.2}

\definecolor{MBlue}{RGB}{0, 80, 180}

\begin{document}
\subsection*{Lists, Permutations, Subsets}

\vspace*{0.2in}\noindent
\textbf{1.} List all the functions from the three-element set $\{1, 2, 3\}$
to the set $\{a, b\}$. Which functions, if any, are one-to-one? Which functions,
if any, are onto?
Do the same for all functions $\{1, 2\} \to \{a, b, c\}$.

\vspace*{0.3in}\noindent
\textbf{2.} Suppose that $|S| = s$ and $|T| = t$. How many functions $S \to T$
are there?

\vspace*{0.3in}\noindent
\textbf{3.} Assuming we pass out all the fruit, if $k \leq n$, in how many ways
can we pass out $k$ identical pieces of fruit to $n$ children if each child may
get at most one? What if $k > n$?
Now answer the same question, but suppose each piece of fruit is distinct.

\vspace*{0.3in}\noindent
\textbf{4.} List in lexicographic (like alphabetic order, so for instance
$\{1,3,5\}$, $\{1,5,3\}$, $\{3,1,5\}$ are in lexicographic order) order all
three-element permutations of the five-element set $\{1, 2, 3, 4, 5\}$.
Note that this problem is probably much easier to do by writing a program to
do the listing and print the permutation for you.
Underline those elements that correspond to the set $\{1, 3, 5\}$.
Draw a rectangle around those that correspond to the set $\{2, 4, 5\}$.
How many three-element permutations of $\{1, 2, 3, 4, 5\}$ correspond to a
given three-element set? How many three-element subsets does the set
$\{1, 2, 3, 4, 5\}$ have?

\iffalse
    12453
    12534
    13425
    13542
    14235
    14352 +
    15243
    15324 +
    23145
    24315
    25341
    31245
    32415
    32541 -
    41325
    42135
    42351
    51342
    52143 -
    52314

    So there are two of each, there are twenty in this list, so there are 10 3-element subsets.
\fi

\vspace*{0.3in}\noindent
\textbf{5.} In how many ways can a class of 20 students choose a group of three
students from among themselves to go to the professor to explain
that the 3-hour labs actually take 10 hours?

\vspace*{0.3in}\noindent
\textbf{6.} Suppose you are organizing a panel discussion on allowing alcohol
on campus. Participants will sit behind a table in the order in which
you list them. You must choose four administrators from a group
of 10 and four students from a group of 20. If the administrators
must sit together in a group and the students must sit together in a
group, in how many ways can you choose and list the eight people?
If you must alternate students and administrators, in how many
ways can you choose and list them?

\vspace*{0.3in}\noindent
\textbf{7.} A basketball team has 12 players. However, only five players play
at any given time during a game. In how may ways can the coach
choose the five players? To be more realistic, the five players playing
a game normally consist of two guards, two forwards, and one center.
If there are five guards, four forwards, and three centers on the team,
in how many ways can the coach choose two guards, two forwards,
and one center? What if one of the centers is equally skilled
at playing forward?

\clearpage\pagebreak\noindent
\textbf{8.} Explain why a function from an $n$-element set to an $n$-element
set is injective if and only if it is surjective. Is this true for sets
with infinite cardinality? Why or why not?

\vspace*{0.3in}\noindent
\textbf{9.} The function $g$ is called an \emph{inverse} to the function $f$
if the domain of $g$ is the range of $f$, if $g(f(x)) = x$ for every $x$ in
the domain of $f$, and if $f(g(y)) = y$ for each $y$ in the range of $f$.
\begin{enumerate}[label=\textbf{\alph*.}]
    \item Explain why a function is a bijection if and only if it has an
          inverse function.
    \item Explain why a function that has an inverse function has only
          one inverse function.
\end{enumerate}



\end{document}