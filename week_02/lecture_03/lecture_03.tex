\documentclass[12pt, letterpaper]{article}

\usepackage[letterpaper,left=0.7in,right=0.7in,top=\dimexpr15mm+1.5\baselineskip,bottom=0.7in]{geometry}
\usepackage{amsmath}
\usepackage{amsthm}
%\usepackage{pxfonts}
\usepackage[]{newpxtext}
\usepackage[varg,vvarbb]{newpxmath}
\usepackage{inconsolata}
\usepackage[T1]{fontenc}
\usepackage{float}
\usepackage{mathtools}
\usepackage{mathrsfs}
\usepackage{anyfontsize}

\theoremstyle{definition}
\newtheorem{defn}{Definition}
\newtheorem{propt}{Proposition}
\newtheorem{eg}{Example}
\newtheorem{soln}{Solution}
\newtheorem{lemma}{Lemma}
\newtheorem{theorem}{Theorem}

\newenvironment{prop}[1]{%
    ~\newline
    \begin{minipage}{\linewidth}
    \rule{\textwidth}{2pt}
        \begin{propt}
}
{%
        \end{propt}
    \rule{\textwidth}{2pt}
    \end{minipage}
}

\usepackage{enumitem}
\usepackage{graphicx}
\usepackage{multicol}
\usepackage[table]{xcolor}

\usepackage{listings}
\lstset{
    basicstyle=\ttfamily, % Default font
    numbers=left,              % Location of line numbers
    % numberstyle=\tiny,          % Style of line numbers
    % stepnumber=2,              % Margin between line numbers
    numbersep=5pt,              % Margin between line numbers and text
    tabsize=2,                  % Size of tabs
    extendedchars=true,
    breaklines=true,            % Lines will be wrapped
    keywordstyle=\color{red},
    frame=b,
    % keywordstyle=[1]\textbf,
    % keywordstyle=[2]\textbf,
    % keywordstyle=[3]\textbf,
    % keywordstyle=[4]\textbf,   \sqrt{\sqrt{}}
    stringstyle=\color{white}\ttfamily, % Color of strings
    showspaces=false,
    showtabs=false,
    xleftmargin=17pt,
    %framexbottommargin=0.2in
    framexleftmargin=17pt,
    framexrightmargin=5pt,
    framexbottommargin=4pt,
    % backgroundcolor=\color{lightgray},
    showstringspaces=false
}
%\lstloadlanguages{ % Check documentation for further languages ...
     % [Visual]Basic,
     % Pascal,
     % C,
     % C++,
     % XML,
     % HTML,
     %Java
%}
% \DeclareCaptionFont{blue}{\color{blue}} 

% \captionsetup[lstlisting]{singlelinecheck=false, labelfont={blue}, textfont={blue}}
\iffalse
\usepackage{caption}
\DeclareCaptionFont{white}{\color{white}}
\DeclareCaptionFormat{listing}{\colorbox[cmyk]{0.7, 0.7, 0.7,0.01}{\parbox{0.98\textwidth}{\hspace{15pt}#1#2#3}}}
\captionsetup[lstlisting]{
    format=listing, 
    labelfont=white, 
    labelformat=empty,
    textfont=white, 
    singlelinecheck=false, 
    %margin=0pt, 
    font={bf}
    }
\fi

\usepackage{fancyhdr}
\setlength{\headheight}{15.2pt}
\pagestyle{fancyplain}
\fancyhf{}
\lhead[Discrete Structures]{Discrete Structures}
\chead[Lecture 3 Notes]{Lecture 3 Notes}
%\rhead[\today]{\today}

\newcommand{\gap}[1][0.3in]{
    \vspace*{#1}
    \noindent\noindent
}

\newcommand{\septext}[1][or]{
    \qquad \text{#1} \qquad
}

\newcommand{\septextor}[0]{
    \septext
}

\newcommand{\septextand}[0]{
    \septext[and]
}

\newcommand{\separrow}[0]{
    \qquad \Rightarrow \qquad
}

\newcommand{\fallingfactorial}[1]{%
  ^{\mspace{2mu}\underline{\mspace{-2mu}#1\mspace{-2mu}}\mspace{2mu}}%
}

\usepackage{circuitikz}
\usepackage{pgfplots}
\pgfplotsset{compat=1.18}
%\usepgfplotslibrary{polar}
\usepackage{wrapfig}
\usepackage{setspace}
\usepackage[export]{adjustbox}

\usepackage{tikzsymbols}

\DeclareMathOperator{\arcsinh}{arcsinh}
\DeclareMathOperator{\Ima}{Im}

\linespread{1.2}

\definecolor{MBlue}{RGB}{0, 80, 180}

\begin{document}
\subsection*{Warm-up}

\vspace*{1in}\noindent
Consider the functions defined on the set $\{1, 2, 3, 4, 5\}$ by the rules
\begin{equation*}
    f(x) = x^5 - 15x^4 + 85x^3 - 224x^2 + 268x - 111
    \qquad\text{and}\qquad
    g(x) = x^2 - 6x + 9.
\end{equation*}
Write down the two sets of ordered pairs
\begin{equation*}
    \left\{ \left(x, f(x)\right) \;|\; x \in \{1, 2, 3, 4, 5\} \right\}
    \qquad\text{and}\qquad
    \left\{ \left(x, g(x)\right) \;|\; x \in \{1, 2, 3, 4, 5\} \right\}.
\end{equation*}
Are they the same function or different functions?

\clearpage\pagebreak\noindent
\subsection*{Relations}

\vspace*{0.2in}\noindent
Looking at the two functions defined in the warm-up,
\begin{equation*}
    f(x) = x^5 - 15x^4 + 85x^3 - 224x^2 + 268x - 111
    \qquad\text{and}\qquad
    g(x) = x^2 - 6x + 9,
\end{equation*}
it makes sense to think that they define different rules because $f$ and $g$
are different. However, the two sets of ordered pairs are the same, and so
these two functions, restricted to $\{1, 2, 3, 4, 5\}$ define the same rule!
Today we will be talking about \emph{relations}. A \emph{relation} from a set
$X$ to a set $Y$ is a set of ordered pairs $(x,y)$ with $x \in X$ and $y \in Y$.
Viewed this way, a function $S \to T$ is simply a relation, $R$, from $S$ to $T$
where every element in $S$ appears as the first element of exactly one ordered
pair in $R$.

However, there are many relations that do not define functions, so that all
functions are relations, but not all relations are functions. Thus relations
are more general than functions. Here are some examples.

\vspace*{0.1in}\noindent
\textbf{Example 1.}
Define a relation, called the \emph{neighbor relation}, on the integers where
$i$ is a neighbor of $j$ is the absolute value of their difference is 1.
The relation has infinite cardinality, but some elements include
\begin{equation*}
    (-1, 0),\;(0, -1),\;(0, 1),\;(1, 0),\;(1, 2),\;(2, 1),\;(2, 3)
\end{equation*}

\vspace*{0.2in}\noindent
\textbf{Example 2.}
When we derived the formula for binomial coefficients, $\binom{n}{k}$, we saw
that there were $k!$ different permutations of a $k$-element subset of an
$n$-element set $S$. Any of these permutations is equivalent for the purposes
of specifying the subset. We can define two permutations to be
\emph{set-equivalent} if they are permutations of the same subset of $S$.
This is a relation on the set of $k$-element permutations of $X$.

\vspace*{0.2in}\noindent
\textbf{Example 3.}
If $S$ and $T$ are sets, then define a relation where the ordered pair $(S, T)$
is an element if and only if $S$ is a subset of $T$. In this case, if $S$ and
$T$ are related, then we usually write $S \subseteq T$.

\vspace*{0.2in}\noindent
\textbf{Example 4.}
If $x, y \in \mathbb{R}$, then define a relation where the ordered pair $(x, y)$
is an element if and only if $x$ is less than $y$. In this case, if $x$ and
$y$ are related, then we usually write $x < y$.

\vspace*{0.2in}
These last examples bring up a good point. We usually write $x<y$, not
$(x,y) \in <$, and $S \subseteq T$, not $(S,T) \in \subseteq$. Likewise, given
a relation $R$, we commonly write $aRb$ if $(a,b) \in R$.

We could define any number of properties of relations, but there are three
(well, four) that are particularly useful, namely \emph{reflexivity},
\emph{(anti)symmetry}, and \emph{transitivity}.

\clearpage\pagebreak\noindent
Let $R$ be a relation defined on a set $X$.
\begin{enumerate}[label=\textbf{\alph*.}]
    \item $R$ is \emph{reflexive} if for all $x \in X$ we have $(x,x) \in R$,
          \emph{i.e.}, $xRx$.
    \item $R$ is \emph{symmetric} if $(x_{1}, x_{2}) \in R$ if and only if
          $(x_{2}, x_{1}) \in R$, \emph{i.e.}, $x_{1}Rx_{2}$ iff $x_{2}Rx_{1}$.
          $R$ is \emph{antisymmetric} if $(x_{1}, x_{2}) \in R$ and
          $(x_{2}, x_{1}) \in R$ only if $x_{1} = x_{2}$, \emph{i.e.},
          $x_{1}Rx_{2}$ and $x_{2}Rx_{1}$ only if $x_{1} = x_{2}$.
    \item $R$ is \emph{transitive} if $(a,b) \in R$ and $(b,c) \in R$ implies
          that $(a,c) \in R$, \emph{i.e.}, $aRb$ and $bRc$ $\Rightarrow$ $aRc$.
\end{enumerate}

\vspace*{0.2in}\noindent
\textbf{Example 5.}
Consider the relations from Examples 1-4. Which are reflexive? Symmetric? Antisymmetric? Transitive?

\iffalse
    Set-equivalence:
    R because a list is a permutation of itself by def,
    S since lists are the same up to permutations,
    T is obvious.

    Neighbor:
    ~R since every number is 0 from itself,
    S because of absolute value,
    ~T since that's 2

    Subset:
    R since $S \subseteq S$ for any $S$
    A since $S \subseteq T$ and $T \subseteq S$ means $S = T$
    T since $S \subseteq T$ and $T \subseteq U$ means $S \subset U$

    Less-than:
    ~R
    ~S, ~A
    T
\fi

\vspace*{0.2in}\noindent
\textbf{Example 6.}
Which properties does the \emph{less than or equal to} relation have?

\vspace*{0.2in}\noindent
\textbf{Example 7.}
Analyze the relations "is a sibling of", "is the mother of", and "is a descendant of".

\vspace*{0.2in}\noindent
\textbf{Example 8.}
Write down the set of all 3-element permutations of $\{1,2,3,4\}$ that are set-equivalent to
\begin{multicols}{4}
    \begin{enumerate}[label=\textbf{\alph*.}]
        \item 123
        \item 124
        \item 134
        \item 234
    \end{enumerate}
\end{multicols}
Do any of these sets have any permutations in common?

\vspace*{0.2in}\noindent
Writing out the set-equivalences of each permutation gives

\begin{center}
    \begin{tabular}{l l}
        123:\qquad & $\{ 123,\; 132,\; 213,\; 231,\; 312,\; 321 \}$ \\
        124:\qquad & $\{ 124,\; 142,\; 214,\; 241,\; 412,\; 421 \}$ \\
        134:\qquad & $\{ 134,\; 143,\; 314,\; 341,\; 413,\; 431 \}$ \\
        234:\qquad & $\{ 234,\; 243,\; 324,\; 342,\; 423,\; 432 \}$
    \end{tabular}
\end{center}

\noindent A simple examination shows that these sets are mutually disjoint.
A careful check will also show that these sets contain every possible
3-element permutation of $\{ 1, 2, 3, 4 \}$.

\vspace*{0.2in}\noindent
\textbf{Example 9.}
Write down the set of all neighbors in $\mathbb{Z}$ of
\begin{multicols}{4}
    \begin{enumerate}[label=\textbf{\alph*.}]
        \item 0
        \item 1
        \item 2
        \item 3
    \end{enumerate}
\end{multicols}
Do any of these sets have any elements in common?

\vspace*{0.2in}\noindent
\textbf{Example 10.}
Write down all subsets of each of the following sets.
\begin{multicols}{3}
    \begin{enumerate}[label=\textbf{\alph*.}]
        \item $\{ 1, 2, 3 \}$
        \item $\{ 1, 2 \}$
        \item $\{ 1, 3 \}$
    \end{enumerate}
\end{multicols}
Do any of these sets of subsets have any elements in common?

\clearpage\pagebreak\noindent
\textbf{Example 11.}
Write down the set of all positive integers less than
\begin{multicols}{3}
    \begin{enumerate}[label=\textbf{\alph*.}]
        \item 2
        \item 3
        \item 4
    \end{enumerate}
\end{multicols}
Do any of these sets have any elements in common?

\vspace*{0.3in}
Note that only Example 8 gives us a scenario that partitions the space (set) of
all things (3-element permutations of $\{ 1, 2, 3, 4 \}$) into mutually disjoint
sets. Indeed, we say that a set $P$ containing subsets of a set $S$ is a
\emph{partition} if its elements are mutually disjoint and their union is $S$.
In this case we call these mutually disjoint sets \emph{equivalence classes},
or just \emph{classes}, and we say that the relation is an
\emph{equivalence relation}. Note also that the relation in Example 8, the
set-equivalence relation, was reflective, symmetric, and transitive. Any
relation that has all three properties is an equivalence relation! It is easy to
prove that these two ideas are the same, but we skip the proof here. Stated as a
proposition, we have

\begin{prop}
    \textbf{(Equivalence Relations from Properties)}
    Let $R$ be a relation on $S$. The following are equivalent:
    \begin{enumerate}[label=\textbf{\alph*.}]
        \item $R$ partitions $S$ into mutually disjoint subsets
              ($R$ is an equivalence relation). That is, for each
              pair of elements $x$ and $y$ of $S$, the sets
              \begin{equation*}
                  S_{x} = \{ z \in S \;|\; (x,z) \in R \}
                  \qquad\text{and}\qquad
                  S_{y} = \{ z \in S \;|\; (y,z) \in R \}
              \end{equation*}
              are either equal or disjoint. These sets are called \emph{equivalence classes}.
        \item $R$ is reflexive, symmetric, and transitive.
    \end{enumerate}
\end{prop}

\vspace*{0.1in}\noindent
We can even go a little further. Suppose we have a set $S$ and we partition
it. Then we can define a relation $R$ on this partition to get an equivalence
relation.

\begin{prop}
    \textbf{(Equivalence Relations from Partitions)}
    Let $P$ be a partition of a set $S$. Suppose we define a relation $R$ by
    \begin{equation*}
        R = \{ (x,y) \;|\; x \;\text{and}\; y \;\text{are in the same element of $P$} \}.
    \end{equation*}
    Then $R$ is an equivalence relation whose equivalence classes are the
    elements of $P$.
\end{prop}

\clearpage\pagebreak\noindent
Let's consider three relations, $<$, $\leq$, and $\subseteq$. We have that $<$
is not relfexive since, \emph{e.g.}, $3 \not{<} 3$, is neither symmetric nor
antisymmetric, but is transitive. The relations $\leq$ and $\subseteq$, on the
other hand, are reflexive, antisymmetric, and transitive. Any such relation is
said to be a \emph{partial order}, and a set that admits a partial order is
called a \emph{partially ordered set} or just \emph{poset}.

We use the word "partial" because there may be elements that cannot be ordered.
For two real numbers $x$ and $y$, either $x \leq y$ or $y \leq x$ (or both, if $x=y$);
however, if $S = \{ 1, 2 \}$ and $T = \{ 1, 3 \}$, then neither $S \subseteq T$
nor $T \subseteq S$, even though $\subseteq$ is a partial order.

Now suppose that $R$ is a partial order on $S$ and $a, b, \in S$.
if either $aRb$ or $bRa$, then we say that $a$ and $b$ are \emph{comparable}.
If neither is true, then they are \emph{incomparable}. If every such pair
of elements in $S$ is comparable, then we say that $R$ is a \emph{total order}
on $S$ and that $S$ is a \emph{totally ordered set}. Thus $\leq$ is a total
order and $\subseteq$ is not.

Sometimes, a totally ordered set has a smallest element. The set
\begin{equation*}
    \{ n \in \mathbb{Z} \;|\; 0 < n \}
\end{equation*}
is totally ordered by $\leq$ and has a smallest element, namely 0. However,
the set
\begin{equation*}
    \{ x \in \mathbb{R} \;|\; 0 < x \}
\end{equation*}
is likewise totally ordered by $\leq$, but it has no smallest element.
A \emph{well-ordered set} is a totally ordered set with the property that
every non-empty subset has a smallest element. This is an important property
with many far-reaching implications.



\end{document}