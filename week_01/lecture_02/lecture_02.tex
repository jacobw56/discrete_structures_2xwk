\documentclass[12pt, letterpaper]{article}

\usepackage[letterpaper,left=0.7in,right=0.7in,top=\dimexpr15mm+1.5\baselineskip,bottom=0.7in]{geometry}
\usepackage{amsmath}
\usepackage{amsthm}
%\usepackage{pxfonts}
\usepackage[]{newpxtext}
\usepackage[varg,vvarbb]{newpxmath}
\usepackage{inconsolata}
\usepackage[T1]{fontenc}
\usepackage{float}
\usepackage{mathtools}
\usepackage{mathrsfs}
\usepackage{anyfontsize}

\theoremstyle{definition}
\newtheorem{defn}{Definition}
\newtheorem{propt}{Proposition}
\newtheorem{eg}{Example}
\newtheorem{soln}{Solution}
\newtheorem{lemma}{Lemma}
\newtheorem{theorem}{Theorem}

\newenvironment{prop}[1]{%
    \vspace*{0.2in}
    ~\newline\noindent
    \begin{minipage}{\linewidth}
    \rule{\textwidth}{2pt}
        \begin{propt}
}
{%
        \end{propt}
    \rule{\textwidth}{2pt}
    \end{minipage}
    \vspace*{0.2in}
    \linebreak
}

\usepackage{enumitem}
\usepackage{graphicx}
\usepackage{multicol}
\usepackage[table]{xcolor}

\usepackage{listings}
\lstset{
    basicstyle=\ttfamily, % Default font
    numbers=left,              % Location of line numbers
    % numberstyle=\tiny,          % Style of line numbers
    % stepnumber=2,              % Margin between line numbers
    numbersep=5pt,              % Margin between line numbers and text
    tabsize=2,                  % Size of tabs
    extendedchars=true,
    breaklines=true,            % Lines will be wrapped
    keywordstyle=\color{red},
    frame=b,
    % keywordstyle=[1]\textbf,
    % keywordstyle=[2]\textbf,
    % keywordstyle=[3]\textbf,
    % keywordstyle=[4]\textbf,   \sqrt{\sqrt{}}
    stringstyle=\color{white}\ttfamily, % Color of strings
    showspaces=false,
    showtabs=false,
    xleftmargin=17pt,
    %framexbottommargin=0.2in
    framexleftmargin=17pt,
    framexrightmargin=5pt,
    framexbottommargin=4pt,
    % backgroundcolor=\color{lightgray},
    showstringspaces=false
}
%\lstloadlanguages{ % Check documentation for further languages ...
     % [Visual]Basic,
     % Pascal,
     % C,
     % C++,
     % XML,
     % HTML,
     %Java
%}
% \DeclareCaptionFont{blue}{\color{blue}} 

% \captionsetup[lstlisting]{singlelinecheck=false, labelfont={blue}, textfont={blue}}
\usepackage{caption}
\DeclareCaptionFont{white}{\color{white}}
\DeclareCaptionFormat{listing}{\colorbox[cmyk]{0.7, 0.7, 0.7,0.01}{\parbox{0.98\textwidth}{\hspace{15pt}#1#2#3}}}
\captionsetup[lstlisting]{
    format=listing, 
    labelfont=white, 
    labelformat=empty,
    textfont=white, 
    singlelinecheck=false, 
    %margin=0pt, 
    font={bf}
    }

\usepackage{fancyhdr}
\setlength{\headheight}{15.2pt}
\pagestyle{fancyplain}
\fancyhf{}
\lhead[Discrete Structures]{Discrete Structures}
\chead[Lecture 1 Notes]{Lecture 1 Notes}
%\rhead[\today]{\today}

\newcommand{\gap}[1][0.3in]{
    \vspace*{#1}
    \noindent\noindent
}

\newcommand{\septext}[1][or]{
    \qquad \text{#1} \qquad
}

\newcommand{\septextor}[0]{
    \septext
}

\newcommand{\septextand}[0]{
    \septext[and]
}

\newcommand{\separrow}[0]{
    \qquad \Rightarrow \qquad
}

\newcommand{\fallingfactorial}[1]{%
  ^{\mspace{2mu}\underline{\mspace{-2mu}#1\mspace{-2mu}}\mspace{2mu}}%
}

\usepackage{circuitikz}
\usepackage{pgfplots}
\pgfplotsset{compat=1.18}
%\usepgfplotslibrary{polar}
\usepackage{wrapfig}
\usepackage{setspace}
\usepackage[export]{adjustbox}

\usepackage{tikzsymbols}

\DeclareMathOperator{\arcsinh}{arcsinh}
\DeclareMathOperator{\Ima}{Im}

\linespread{1.2}

\definecolor{MBlue}{RGB}{0, 80, 180}

\begin{document}
\subsection*{Warm-up}

\vspace*{1in}\noindent
In how many ways can you draw a first card and then a second card from a deck
of 52 cards if you \emph{do not} put the drawn cards back into the deck?

\vspace*{0.4in}\noindent
In how many ways can you draw a first card and then a second card from a deck
of 52 cards if you \emph{do} put the drawn cards back into the deck?

\clearpage\pagebreak\noindent
\subsection*{Lists, Permutations, Subsets}

\noindent
\textbf{Example 1.}
A password for a certain computer system is supposed to be between four
and eight characters long and composed of lowercase and/or uppercase
letters, and the digits 0-9. How many passwords are possible? What counting
principles did you use? Estimate the percentage of the possible passwords that
have exactly four letters.


\clearpage\pagebreak\noindent
\subsection*{Lists, Permutations, Subsets}

\noindent
\textbf{Example 1.}
A password for a certain computer system is supposed to be between four
and eight characters long and composed of lowercase and/or uppercase
letters, and the digits 0-9. How many passwords are possible? What counting
principles did you use? Estimate the percentage of the possible passwords that
have exactly four letters.

\vspace*{0.2in}\noindent
First, let's consider whether we can or should use either the Sum or the
Product Principle. We know that we can make a password containing four, five,
six, seven, or eight characters. Let $P_{i}$ for
$i \in \left\{ 4, 5, 6, 7, 8 \right\}$ be the set of all passwords with $i$
characters. Then for $i \neq j$, we have that $P_{i}$ and $P_{j}$ are
disjoint, and if $P$ is the set of all passwords that satisfy the problem, then
by the Sum Principle we have
\begin{equation*}
    |P| = \left|\bigcup_{i=4}^{8}P_{i}\right| = \sum_{i=4}^{8}|P_{i}|.
\end{equation*}
We now need to find $|P_{i}|$. For an $i$-character password, there are
26 choices of both lower and uppercase letters, and ten digits, for a total
of 62 choices of character. By the Product Principle, we have
$|P_{i}| = 62^{i}$. Therefore the total number of passwords is given by
\begin{equation*}
    |P| = \sum_{i=4}^{8}|P_{i}| = 62^4 + 62^5 + 62^6 + 62^7 + 62^8 = 221,919,451,335,856.
\end{equation*}
Of these, $62^4$ of them contain four characters, so the percentage of passwords
with four characters is given by
\begin{equation*}
    \frac{62^4}{62^4 + 62^5 + 62^6 + 62^7 + 62^8} \times 100 \approx 0.0000067 \%
\end{equation*}
Suppose that a computer can test one password every microsecond. Then it would
take
\begin{equation*}
    62^4 * 0.000001 \approx 14.78 s
\end{equation*}
to test every four-character password, but it would take
\begin{equation*}
    62^8 * 0.000001 \approx 218,340,105.58 s
\end{equation*}
to test every eight-character password, or around 7 years!
If we allowed passwords of 20 characters, then it would take this computer about
22 sextillion years ($22 \times 10^{22}$) to try every 20-character password!
For reference, the best science estimates the age of the univserse to be about
13 billion years, some 12 orders of magnitude difference!
Lesson: use longer passwords!

\clearpage\pagebreak\noindent
Notice that we used the Product Principle without writing a union of sets of
equal size. We could have, but it would have been a little clumsy-looking.
Instead we can use the following, second version of the Product Principle,
which is easily derived using mathematical induction, which we will discuss
in detail in a couple of weeks.
\begin{prop}
    \textbf{(Product Principle - Version 2)}
    If $S$ is a set of lists of length $m$ having the properties
    \begin{enumerate}
        \item there are $i_{1}$ different first elements of lists in $S$, and
        \item for each $k>1$ and each choice of the first $k-1$ elements of a
              list in $S$, there are $i_{k}$ choices of elements in position $k$ of
              those lists,
    \end{enumerate}
    then there are
    \begin{equation*}
        \prod_{k=1}^{m}i_{k} = i_{1} i_{2} i_{3} \cdots i_{m}
    \end{equation*}
    lists in $S$.
\end{prop}
Let's apply this version of the product principle to compute the number
of $m$-character passwords. Because an $m$-character password is simply a list
of $m$ characters and because there are 62 different first elements of the
password and 62 choices for each other position of the password, we have
that $i_{1} = 62$, $i_{2} = 62$, $\dots$ , $i_{m} = 62$. Thus, this version of
the Product Principle tells us immediately that the number of passwords of
length $m$ is
\begin{equation*}
    \prod_{k=1}^{m}i_{k} = i_{1} i_{2} i_{3} \cdots i_{m} = 62^{m}.
\end{equation*}

\clearpage\pagebreak\noindent
We very conveniently used the word \emph{list} above without ever defining it.
Let's clear this up now, but first we will define a function and some
associated terms so that our definition will be precise.

A \emph{function}, $f$, from a set $S$ to a set $T$, written
\begin{equation*}
    f : S \to T
\end{equation*}
is a mapping that associates exactly one element of $T$ to each element of $S$.
The set $S$ is called the \emph{domain} of $f$ and the set $T$ is called its
\emph{range} or \emph{codomain}. We often use the notation $f(x)$ so indicate
that the function $f$ is acting on some element $x \in S$. Sometimes we refer
to this mapping, emphasizing that it is associating an element $x \in S$ to
and element $y \in T$ using the notation
\begin{equation*}
    x \mapsto y
\end{equation*}
It is often handy to know that the \emph{image} of a function $f : S \to T$,
denoted $\Ima f$, is the set of all values $f(x)$ for $x \in S$, \emph{i.e.},
\begin{equation*}
    \Ima f = \left\{ f(x) \;|\; x \in S \right\}.
\end{equation*}

In this course we often explore functions over finite sets, in which case
it is sometimes possible to simply explicitly list every association between
our domain and codomain; \emph{e.g.}
\begin{eqnarray}
    f(\alpha) = \text{Tim}, f(\beta) = \text{\LARGE\WorstTree}, f(\gamma) = 7
\end{eqnarray}
defines a function
\begin{equation*}
    f : \left\{ \alpha, \beta, \gamma \right\} \to \left\{ \text{Tim}, \text{\LARGE\WorstTree}, 7 \right\}.
\end{equation*}
We could also write the function as a set of \emph{ordered pairs}; \emph{i.e.},
\begin{equation*}
    \left\{ \left(\alpha, \text{Tim}\right), \left( \beta, \text{\LARGE\WorstTree} \right), \left( \gamma, 7 \right) \right\}.
\end{equation*}
Now we can give a precise definition of a \emph{list}.

A \emph{list} of $k$ elements from a set $T$ is a function
\begin{equation*}
    \left\{ 1, 2, \dots, k \right\} \to T.
\end{equation*}

\vspace*{0.1in}\noindent
\textbf{Example 2.}
How many functions are there from a two-element set to a three-element set?
How many functions are there from a three-element set to a two-element set?


\clearpage\pagebreak\noindent
A function $f$ is \emph{one-to-one}, or \emph{injective}, or an \emph{injection},
if $f(x) \neq f(y)$ when $x \neq y$. In symbols, $f$ is an \emph{injection} if
\begin{equation*}
    x \neq y \quad\Rightarrow\quad f(x) \neq f(y).
\end{equation*}
Note that this is equivalent to
\begin{equation*}
    f(x) = f(y) \quad\Rightarrow\quad x = y
\end{equation*}
by \emph{contraposition}, the veracity of which is proved by \emph{modus tollens}
or \emph{DeMorgan's laws}. Note also that a function is injective if and only if
it passes the horizontal line test.

A function, $f : S \to T$ is called onto, or \emph{surjective}, or a \emph{surjection},
if for all $y \in T$ there is an $x \in S$ such that $f(x) = y$. Note that,
for example, $x \mapsto x^2$ is not surjective as a function
$\mathbb{R} \to \mathbb{R}$, but it is a surjection as a function
$\mathbb{R} \to [ 0, \infty )$. Indeed, every function is a surjection onto
its image; \emph{i.e.},
\begin{equation*}
    f : S \to \Ima f
\end{equation*}
is always surjective.

A function that is both injective and surjective is called \emph{bijective},
or is a \emph{bijection}, or a \emph{one-to-one correspondence}. A bijection
from a set to itself is called a \emph{permutation}. We also have the
\emph{Bijection Principle}:
\begin{prop}
    \textbf{(Bijection Principle)}
    If $f : S \to T$ is a bijection, then $|S| = |T|$.
\end{prop}


\vspace*{0.2in}\noindent
\textbf{Example 3.}
Using two- or three-element sets as domains and ranges, find an example
of a one-to-one function that is not onto.
Using two- or three-element sets as domains and ranges, find an example
of an onto function that is not one-to-one.

\clearpage\pagebreak\noindent
\textbf{Example 4.}
Among all iterations of line 5 of the pseudocode, what is the total number
of times this line checks three points to see if they are collinear?

\begin{center}
    \begin{lstlisting}[label=ex4, caption={Example 4 Code}]
    trianglecount = 0
    for i = 1 to n-1
        for j = i+1 to n
            for k = j+1 to n
                if i, j, and k are not collinear
                    trianglecount += 1
    \end{lstlisting}
\end{center}

Note that, since $i < j < k$, the code examines each such increasing triple
$(i, j, k)$ exactly once. If $n = 4$, then we would count the increasing triples
$(1,2,3)$, $(1,2,4)$, $(1,3,4)$, and $(2,3,4)$. We could count all such triples,
but note instead that this is precisely the same as the number of 3-element
subsets of $\left\{ 1, 2, \dots, n \right\}$. Because of the Bijection Principle,
we can count such subsets instead to find \texttt{trianglecount}.

Since sets do not have repeating elements, if we want to choose $k$ elements
from a set of cardinality $n$, with $k \leq n$, then we have $n$ choices
for the first element, $n-1$ for the second, up to $n-k+1$ choices for the
$k$th element. This value, called the $k$th \emph{falling factorial power of n}
is important enough that Donald E. Knuth (a name you ought to learn now)
suggested it have its own symbol, $n \fallingfactorial{k}$. We have
\begin{equation*}
    n\fallingfactorial{k} = \prod_{i=0}^{k-1}(n-i) = n(n-1)\cdots(n-k+1) = \frac{n!}{(n-k)!}.
\end{equation*}

We still have a problem! The subsets we have generated have repeats!
For instance, if $n=4$ using the above logic to choose 3-element subsets would
yield $\{1,2,3\}$, $\{2,3,1\}$, $\{3,1,2\}$, $\{3,2,1\}$, $\{2,1,3\}$, and $\{1,3,2\}$,
even though these are all the same set; that is, we get 6 different lists,
which each represent the same set. Indeed, there are 6 such lists
because there are three different numbers in each and $3 \cdot 1 \cdot 1 = 6$
ways to choose them. This argument gives us our general solution. If we wish
to choose $k$ unique elements from an $n$-element set where order does not matter,
then we have
\begin{equation*}
    \binom{n}{k} = \frac{n\fallingfactorial{k}}{k!} = \frac{n!}{k!(n-k)!}.
\end{equation*}
The numbers $\binom{n}{k}$, read \emph{n choose k}, are called
\emph{binomail coefficients}, for reasons you will explore in the weekly project.



\end{document}