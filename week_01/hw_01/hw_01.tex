\documentclass[12pt, letterpaper]{article}

\usepackage[letterpaper,left=0.7in,right=0.7in,top=\dimexpr15mm+1.5\baselineskip,bottom=0.7in]{geometry}
\usepackage{amsmath}
\usepackage{amsthm}
%\usepackage{pxfonts}
\usepackage[]{newpxtext}
\usepackage[varg,vvarbb]{newpxmath}
\usepackage{inconsolata}
\usepackage[T1]{fontenc}
\usepackage{float}
\usepackage{mathtools}
\usepackage{mathrsfs}
\usepackage{anyfontsize}

\theoremstyle{definition}
\newtheorem{defn}{Definition}
\newtheorem{propt}{Proposition}
\newtheorem{eg}{Example}
\newtheorem{soln}{Solution}
\newtheorem{lemma}{Lemma}
\newtheorem{theorem}{Theorem}

\newenvironment{prop}[1]{%
    \vspace*{0.2in}
    \linebreak
    \begin{minipage}{\linewidth}
    \rule{\textwidth}{2pt}
        \begin{propt}
}
{%
        \end{propt}
    \rule{\textwidth}{2pt}
    \end{minipage}
    \vspace*{0.2in}
    \linebreak
}

\usepackage{enumitem}
\usepackage{graphicx}
\usepackage{multicol}
\usepackage[table]{xcolor}

\usepackage{listings}
\lstset{
    basicstyle=\ttfamily, % Default font
    numbers=left,              % Location of line numbers
    % numberstyle=\tiny,          % Style of line numbers
    % stepnumber=2,              % Margin between line numbers
    numbersep=5pt,              % Margin between line numbers and text
    tabsize=2,                  % Size of tabs
    extendedchars=true,
    breaklines=true,            % Lines will be wrapped
    keywordstyle=\color{red},
    frame=b,
    % keywordstyle=[1]\textbf,
    % keywordstyle=[2]\textbf,
    % keywordstyle=[3]\textbf,
    % keywordstyle=[4]\textbf,   \sqrt{\sqrt{}}
    stringstyle=\color{white}\ttfamily, % Color of strings
    showspaces=false,
    showtabs=false,
    xleftmargin=17pt,
    %framexbottommargin=0.2in
    framexleftmargin=17pt,
    framexrightmargin=5pt,
    framexbottommargin=4pt,
    % backgroundcolor=\color{lightgray},
    showstringspaces=false
}
%\lstloadlanguages{ % Check documentation for further languages ...
     % [Visual]Basic,
     % Pascal,
     % C,
     % C++,
     % XML,
     % HTML,
     %Java
%}
% \DeclareCaptionFont{blue}{\color{blue}} 

% \captionsetup[lstlisting]{singlelinecheck=false, labelfont={blue}, textfont={blue}}
\usepackage{caption}
\DeclareCaptionFont{white}{\color{white}}
\DeclareCaptionFormat{listing}{\colorbox[cmyk]{0.7, 0.7, 0.7,0.01}{\parbox{0.98\textwidth}{\hspace{15pt}#1#2#3}}}
\captionsetup[lstlisting]{
    format=listing, 
    labelfont=white, 
    labelformat=empty,
    textfont=white, 
    singlelinecheck=false, 
    %margin=0pt, 
    font={bf}
    }

\usepackage{fancyhdr}
\setlength{\headheight}{15.2pt}
\pagestyle{fancyplain}
\fancyhf{}
\lhead[Discrete Structures]{Discrete Structures}
\chead[Homework 1]{Homework 1}
%\rhead[\today]{\today}

\newcommand{\gap}[1][0.3in]{
    \vspace*{#1}
    \noindent\noindent
}

\newcommand{\septext}[1][or]{
    \qquad \text{#1} \qquad
}

\newcommand{\septextor}[0]{
    \septext
}

\newcommand{\septextand}[0]{
    \septext[and]
}

\newcommand{\separrow}[0]{
    \qquad \Rightarrow \qquad
}

\usepackage{circuitikz}
\usepackage{pgfplots}
\pgfplotsset{compat=1.18}
%\usepgfplotslibrary{polar}
\usepackage{wrapfig}
\usepackage{setspace}
\usepackage[export]{adjustbox}

\DeclareMathOperator{\arcsinh}{arcsinh}

\linespread{1.2}

\definecolor{MBlue}{RGB}{0, 80, 180}

\begin{document}
\subsection*{Counting: The Sum and Product Principles}

\vspace*{0.3in}\noindent
\textbf{1.} Consider the following code.
\begin{center}
    \begin{lstlisting}[label=p1, caption={Problem 1 Code}, mathescape=true]
    for i = 2 to n
        j = i
        while (j $\geq$ 2) and (A[j] < A[j-1])
            swap A[j] and A[j-1]
            j = j-1
    \end{lstlisting}
\end{center}
What is the maximum number of times (considering all lists of $n$ items that
you could be asked to sort) the program makes the comparison $A[j] < A[j - 1]$?
Describe as succinctly as you can those lists that require this number of
comparisons.

\vspace*{0.3in}\noindent
\textbf{2.} Five schools are going to send their badminton teams to a tournament
in which each team must play each other team exactly once. How many games are
required?

\vspace*{0.3in}\noindent
\textbf{3.} In how many ways can a 10-person club select a president and a
secretary-treasurer from among its members (assuming that a member cannot fill
both positions)?

\vspace*{0.3in}\noindent
\textbf{4.} In how many ways can a 10-person club select a two-person executive
committee from among its members?

\vspace*{0.3in}\noindent
\textbf{5.} In how many ways can a 10-person club select a president and a
two-person executive advisory board from among its members (assuming that the
president is not on the advisory board)?

\vspace*{0.3in}\noindent
\textbf{6.} Using the formula for $\binom{n}{2}$, it is straightforward to show
that
\begin{equation*}
    n\binom{n-1}{2} = \binom{n}{2}(n-2).
\end{equation*}
However, this proof simply uses blind substitution and simplification.
Find a more conceptual explanation of why this formula is true.
(Hint: Think in terms of officers and committees in a club.)





\end{document}